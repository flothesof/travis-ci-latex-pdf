% !TEX encoding = UTF-8 Unicode
%% start of file `template.tex'.
%% Copyright 2006-2013 Xavier Danaux (xdanaux@gmail.com).
%
% This work may be distributed and/or modified under the
% conditions of the LaTeX Project Public License version 1.3c,
% available at http://www.latex-project.org/lppl/.


\documentclass[11pt,a4paper,sans]{moderncv}        % possible options include font size ('10pt', '11pt' and '12pt'), paper size ('a4paper', 'letterpaper', 'a5paper', 'legalpaper', 'executivepaper' and 'landscape') and font family ('sans' and 'roman')

% moderncv themes
\moderncvstyle{casual}                             % style options are 'casual' (default), 'classic', 'oldstyle' and 'banking'
\moderncvcolor{green}                               % color options 'blue' (default), 'orange', 'green', 'red', 'purple', 'grey' and 'black'
%\renewcommand{\familydefault}{\sfdefault}         % to set the default font; use '\sfdefault' for the default sans serif font, '\rmdefault' for the default roman one, or any tex font name
%\nopagenumbers{}                                  % uncomment to suppress automatic page numbering for CVs longer than one page

% character encoding
\usepackage[utf8]{inputenc}                       % if you are not using xelatex ou lualatex, replace by the encoding you are using
\usepackage{savesym}
\savesymbol{fax}
\usepackage{marvosym}
\restoresymbol{MARV}{fax} % fix symbol error http://tex.stackexchange.com/questions/146138/the-package-marvosym-seems-to-clash-with-the-moderncv-class-what-can-i-do

\DeclareUnicodeCharacter{20AC}{\EUR{}}
%\usepackage{CJKutf8}                              % if you need to use CJK to typeset your resume in Chinese, Japanese or Korean

% adjust the page margins
\usepackage[scale=0.75]{geometry}
%\setlength{\hintscolumnwidth}{3cm}                % if you want to change the width of the column with the dates
%\setlength{\makecvtitlenamewidth}{10cm}           % for the 'classic' style, if you want to force the width allocated to your name and avoid line breaks. be careful though, the length is normally calculated to avoid any overlap with your personal info; use this at your own typographical risks...

% personal data
\name{Florian}{Le Bourdais}
\title{Research and development engineer in physics with coding skills and a love for data}                               % optional, remove / comment the line if not wanted
\address{129 avenue du général Leclerc}{75014 Paris}{France}% optional, remove / comment the line if not wanted; the "postcode city" and "country" arguments can be omitted or provided empty
\phone[mobile]{+33~(6)~41 67 73 42}                   % optional, remove / comment the line if not wanted; the optional "type" of the phone can be "mobile" (default), "fixed" or "fax"
\email{florian.s.lebourdais@gmail.com}                               % optional, remove / comment the line if not wanted
\homepage{http://flothesof.github.io}                         % optional, remove / comment the line if not wanted
\social[linkedin]{Florian Le Bourdais}                        % optional, remove / comment the line if not wanted
\social[github]{flothesof}                              % optional, remove / comment the line if not wanted
%\photo[64pt][0.4pt]{picture}                       % optional, remove / comment the line if not wanted; '64pt' is the height the picture must be resized to, 0.4pt is the thickness of the frame around it (put it to 0pt for no frame) and 'picture' is the name of the picture file
%\quote{Some quote}                                 % optional, remove / comment the line if not wanted

% to show numerical labels in the bibliography (default is to show no labels); only useful if you make citations in your resume
%\makeatletter
%\renewcommand*{\bibliographyitemlabel}{\@biblabel{\arabic{enumiv}}}
%\makeatother
%\renewcommand*{\bibliographyitemlabel}{[\arabic{enumiv}]}% CONSIDER REPLACING THE ABOVE BY THIS

% bibliography with mutiple entries
%\usepackage{multibib}
%\newcites{book,misc}{{Books},{Others}}
%----------------------------------------------------------------------------------
%            content
%----------------------------------------------------------------------------------
\begin{document}
%\begin{CJK*}{UTF8}{gbsn}                          % to typeset your resume in Chinese using CJK
%-----       resume       ---------------------------------------------------------
\makecvtitle

\section{Experience}
\cventry{2011--present}{Project manager in non destructive evaluation / Research engineer}{CEA (French atomic energy commission) Non-destructive testing department}{Saclay, France}{}{Project manager for 5 year research program for development of innovative inspection techniques for ASTRID fast breeder reactor project.
\begin{itemize}
\item Managed a yearly budget of approximately 1 M€ involving the work of 5+ engineers and researchers.
\item Conducted my own research and presented results in numerous national and international conferences. 
\item Specified new features for the commercial software CIVA developed internally and supervised implementation as well as testing.
\end{itemize}}
\cventry{2010--2011}{Research engineer}{CEA-JAEA (Japanese atomic energy commission)}{Ibaraki, Japan}{}{Facilitated exchanges between a Japanese and a French company over research topics related to sodium-cooled fast reactors.
\begin{itemize}%
\item Worked in a Japanese environment with full cultural integration.
\item Designed and led my own laboratory testing campaign using custom setup ultrasonic transducers.
\item Implemented an ultrasonic reconstruction algorithm and applied it to evaluate the potential of this technique.
\end{itemize}}
\cventry{2008--2009}{Index arbitrage trader}{Calyon Securities}{Tokyo, Japan}{}{Full time trader in charge of a 5 billion Yen portfolio.\newline{}I oversaw the 3-monthly roll of a complex securities and derivatives portfolio hedging against market risks and developing tools to analyze the P'n'L.}

\section{Education}
\cventry{2006--2010}{Master's degree in engineering}{Ecole nationale des Ponts et Chaussées}{Paris, France}{\textit{Specialty: mechanical engineering}}{Classes taken: mechanics (continuum, fluids, acoustics, fracture mechanics), mathematics and computing (finite elements, programming, statistics, variational formulations), humanities (introduction to management and accounting, public speaking, communication in English)}  % arguments 3 to 6 can be left empty
\cventry{2004--2006}{Classes préparatoires}{Centre International de Valbonne, France}{}{}{Subjects: physics, chemistry, engineering, mathematics.}

\section{Master thesis}
\cvitem{title}{\emph{Under-sodium viewing projects at experimental fast reactor Joyo}}
\cvitem{supervisors}{Takafumi Aoyama, Gilles Rodriguez, François Baqué}
\cvitem{description}{Lost fuel pin detection using ultrasonic imaging methods: experimental protocol and analysis of first water tests.}

\section{Languages}
\cvitemwithcomment{French and German}{Mother tongues}{I am a blend of French and German culture.}
\cvitemwithcomment{English}{Fluent, business as well as casual}{Frequent communication in a business setting, TOEIC 985/990, TOEFL iBT 114/120.}
\cvitemwithcomment{Japanese}{Intermediate}{Lived two years in Japan, passed JLPT N3.}

\section{Computer skills}
\cvitem{Python}{Proficient with the scientific stack (IPython Notebook, NumPy, SciPy, scikit-learn, pandas): vector and matrix manipulation, algorithms, machine learning, exploratory data analysis, plotting, user interfaces, webservers.}
\cvitem{Other}{MATLAB (if necessary), C++ (when performance is needed).}
\cvitem{More}{I am proficient in Word, Excel and PowerPoint. I was a VBA expert when I worked in finance.}

\section{Interests}
\cvitem{Blogging}{I maintain a technical blog written using IPython Notebooks at \url{https://flothesof.github.io}. This is a playground for my side projects. In particular, it features blog articles about language learning, graphics, statistics, physics and guitar sound signal processing.}
\cvitem{Online classes}{I am a big fan of MOOCs. I finished Andrew Ng's \emph{Machine learning} class with a grade of 100\% in April 2015. I also took \emph{Learning How to Learn: Powerful mental tools to help you master tough subjects} in February 2015, which I finished with a grade of 99.4\%.}
\cvitem{Coding}{Finalist in the Google France Hashcode 2015 competition.}
\cvitem{Project Euler}{I regularly solve project Euler algorithmic puzzles. These puzzles usually yield with new insights about everything math, from number theory to geometry. I'm currently at 82 of those solved correctly.}
\cvitem{Guitar}{I am an enthusiastic guitar player.}
\cvitem{Japan}{Ever since living in Japan, I am intrigued by the differences and similitudes between western and Japanese culture and language.}
\cvitem{Reviewer}{I have been a reviewer for the book "OpenCV with Python blueprints" by Michael Beyeler. I was approached by the publisher of the book through my blog.}

\section{References}
\cvitem{CEA}{Frédéric Jenson, head of \emph{Industrial Projects} lab at CEA Non-destructive testing department: \href{mailto:frederic.jenson@cea.fr}{frederic.jenson@cea.fr}.}
\cvitem{Calyon Securities}{Pierre Roy, head of arbitrage desk at Millenium Hedge Fund: \href{mailto:pierre.roy@mlp.com}{pierre.roy@mlp.com}.}

% Publications from a BibTeX file without multibib
%  for numerical labels: 
\renewcommand{\bibliographyitemlabel}{\@{\arabic{enumiv}}}% CONSIDER MERGING WITH PREAMBLE PART
%  to redefine the heading string ("Publications"): \renewcommand{\refname}{Articles}
\nocite{*}
\bibliographystyle{plain}
\bibliography{my_publications.bib}                        % 'publications' is the name of a BibTeX file

% Publications from a BibTeX file using the multibib package
%\section{Publications}
%\nocitebook{book1,book2}
%\bibliographystylebook{plain}
%\bibliographybook{publications}                   % 'publications' is the name of a BibTeX file
%\nocitemisc{misc1,misc2,misc3}
%\bibliographystylemisc{plain}
%\bibliographymisc{publications}                   % 'publications' is the name of a BibTeX file

%\clearpage
%%-----       letter       ---------------------------------------------------------
%% recipient data
%\recipient{Company Recruitment team}{Company, Inc.\\123 somestreet\\some city}
%\date{January 01, 1984}
%\opening{Dear Sir or Madam,}
%\closing{Yours faithfully,}
%\enclosure[Attached]{curriculum vit\ae{}}          % use an optional argument to use a string other than "Enclosure", or redefine \enclname
%\makelettertitle
%
%Lorem ipsum dolor sit amet, consectetur adipiscing elit. Duis ullamcorper neque sit amet lectus facilisis sed luctus nisl iaculis. Vivamus at neque arcu, sed tempor quam. Curabitur pharetra tincidunt tincidunt. Morbi volutpat feugiat mauris, quis tempor neque vehicula volutpat. Duis tristique justo vel massa fermentum accumsan. Mauris ante elit, feugiat vestibulum tempor eget, eleifend ac ipsum. Donec scelerisque lobortis ipsum eu vestibulum. Pellentesque vel massa at felis accumsan rhoncus.
%
%Suspendisse commodo, massa eu congue tincidunt, elit mauris pellentesque orci, cursus tempor odio nisl euismod augue. Aliquam adipiscing nibh ut odio sodales et pulvinar tortor laoreet. Mauris a accumsan ligula. Class aptent taciti sociosqu ad litora torquent per conubia nostra, per inceptos himenaeos. Suspendisse vulputate sem vehicula ipsum varius nec tempus dui dapibus. Phasellus et est urna, ut auctor erat. Sed tincidunt odio id odio aliquam mattis. Donec sapien nulla, feugiat eget adipiscing sit amet, lacinia ut dolor. Phasellus tincidunt, leo a fringilla consectetur, felis diam aliquam urna, vitae aliquet lectus orci nec velit. Vivamus dapibus varius blandit.
%
%Duis sit amet magna ante, at sodales diam. Aenean consectetur porta risus et sagittis. Ut interdum, enim varius pellentesque tincidunt, magna libero sodales tortor, ut fermentum nunc metus a ante. Vivamus odio leo, tincidunt eu luctus ut, sollicitudin sit amet metus. Nunc sed orci lectus. Ut sodales magna sed velit volutpat sit amet pulvinar diam venenatis.
%
%Albert Einstein discovered that $e=mc^2$ in 1905.
%
%\[ e=\lim_{n \to \infty} \left(1+\frac{1}{n}\right)^n \]
%
%\makeletterclosing

%\clearpage\end{CJK*}                              % if you are typesetting your resume in Chinese using CJK; the \clearpage is required for fancyhdr to work correctly with CJK, though it kills the page numbering by making \lastpage undefined
\end{document}


%% end of file `template.tex'.
